
% Thesis Abstract -----------------------------------------------------


%\begin{abstractslong}    %uncommenting this line, gives a different abstract heading


\begin{resumen}        %this creates the heading for the abstract page

\selectlanguage{spanish}

%Motivación

La creación de historias de ficción es una tarea de gran complejidad que implica un proceso creativo donde el autor ha de mezclar personajes, conflictos y tramas. Según el afamado escritor Robert McKee,
% CITA http://es.wikipedia.org/wiki/Robert_McKee
el guionista, al igual que el compositor musical, es un artista creativo que comienza con una página en blanco y acaba con una obra.


%Razonamiento

Con el auge de los videojuegos en el mundo del entretenimiento
% CITA REQUERIDA
y especialmente en el caso de los \textit{sandboxes}, o mundos abiertos, se plantea una problemática de difícil solución: El esfuezo requerido para elaborar un buen guión de ficción interactiva es directamente proporcional al número de personajes que deben existir y a su complejidad, por eso es común que en estos tipos de ficción, miles de extras sean virtualmente atrezzo andante
% CITA requerida
. En este campo, disciplinas como la Inteligencia Artificial y la Vida Artificial tienen mucho que aportar.

%Solución propuesta

El presente trabajo pretende abordar esta problemática estableciendo una mecánica que ofrezca subtramas coherentes entre sí e interesantes para todos los personajes del mundo virtual.
Para aportar coherencia a todos los personajes, el sistema se inspira en los clásicos sistemas multiagente
% CITA requerida
. Según esta premisa, cada personaje es modelado como un agente que nace, crece, se relaciona y muere, en un entorno definido con dimensiones espacio-temporales. El mundo virtual es modelado como un sistema auto-organizativo donde cada elemento influye en los demás, mediante interacciones inspiradas en la naturaleza
% CITA requerida
, y aplica mecanismos adaptativos de disminución de la fricción y aumento de la sinergia. De este modo, todo comportamiento tiene unas relaciones causa-efecto que quedan descritas, contextualizadas y explicadas, aportando coherencia al mundo.

De manera general, diseñar un sistema auto-organizativo de tales características es una tarea compleja que ha de asegurar la estabilidad de la sociedad generada y una cierta convergencia, por lo que han de diseñarse mecanismos mediadores que regulen el comportamiento de los agentes, evitando la generación de "sistemas frágiles".
%CITA requerida

Como segundo objetivo, se pretende que el sistema auto-organizativo resultante presente un entorno apto para las historias principales y secundarias de cada obra específica, y además resulte interesante para el espectador / jugador. Para tal fin, se define un conjunto de probabilidades y estados asociados a las acciones de los agentes y, mediante algoritmos multi-objetivo bioinspirados, se buscará la optimización de dichas probabilidades para que emerjan arquetipos literarios según unas reglas y patrones definidos por el creador de la ficción. Los arquetipos, en contraposición a los estereotipos, son comportamientos y patrones universalmente aceptados y presentes a un nivel de subconsciente colectivo, por lo que le permiten al espectador empatizar con los personajes y sumergirse en las tramas. 
% cita requerida (arquetipos jungianos)


%Resultados

Las ideas expuestas a lo largo del presente trabajo han sido implementadas en un prototipo con el que se han podido realizar una serie de experimentos que confirman su base teórica.  





\end{resumen}




%\end{abstractlongs}


% ---------------------------------------------------------------------- 
