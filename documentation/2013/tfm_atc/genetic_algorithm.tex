
\chapter{Algoritmo genético para la búsqueda de arquetipos emergentes}
\label{cha:genetic_algorithm}

Llegados a este punto, disponemos de un sistema auto-organizativo de vida
artificial que simula el paso del tiempo sobre una sociedad, registrando todas
las acciones que realizan sus agentes.

Los agentes disponen de mecanismos para competir por los recursos, cooperar y
tener descendencia. El momento de realización de cada acción, su frecuencia
y efectividad están definidos por las características del agente, que a
su vez están generadas a partir de los valores definidos en el perfil. Todas
las acciones realizadas por los agentes tienen una coherencia respecto a los
demás agentes y el entorno.

El objetivo de este apartado consiste en la búsqueda de arquetipos emergentes.
La ejecución de un entorno MADE es un proceso estocástico: su ejecución es
azarosa pero orientada por el conjunto de parámetros de los perfiles,
que influyen directamente en las decisiones y características de los agentes. A
priori, es difícil predecir qué efecto tiene el aumento o la disminución del
valor de un parámetro específico en el entorno. Tampoco habrá dos 
ejecuciones iguales, así que tendremos que trabajar utilizando cálculos
estadísticos.

Los registros de los agentes pueden llegar a tener unas mil líneas. Si
obtenemos un sistema para procesar esas líneas podríamos evaluar el
comportamiento del agente.

La hipótesis planteada es la siguiente:\\



Utilizando un algoritmo genético donde:

\begin{itemize}
\item Se fijan características del entorno como el tamaño del mapa, el número
inicial de agentes o el número de perfiles.
\item Los genes que forman el cromosoma son cada uno de los parámetros de los
perfiles disponibles para los agentes.
\item La función fitness viene definida por el usuario mediante:
  \begin{itemize}
   \item el uso de expresiones regulares que son capaces de encontrar arquetipos
en los registros de los agentes
   \item una asignación de valor a la población en función del
número de agentes que cumplen un arquetipos
  \end{itemize}
\end{itemize}

  

Seríamos capaces de optimizar los valores de los perfiles de manera que la
ejecución de un Entorno MADE genere agentes con vidas ``interesantes'' según el
modelado hecho por el usuario de arquetipos y patrones.

Por ejemplo: Partiendo del entorno prototipo diseñado con ratas, podríamos
buscar dos contextos diferentes:

Caso 1: el autor desea un contexto de ciencia ficción distrópica:
El contexto deseado podría ser una sociedad que ve mermada su población por la
falta de víveres. Las arquetipos que nos interesarían podrían ser de villanos y
supervivientes, con mucha acción y violencia.

Caso 2: el autor desea contar una soap opera:
El contexto deseado podría ser una sociedad floreciente, donde no hay
excesiva falta de recursos.Podrían premiarse los arquetipos de relaciones
amorosas, la venganza, las experiencias personales.

Ninguno de los arquetipos citados anteriormente han sido modelados de manera
directa en el comportamiento de nuestras ratas.


\section{Parametrización del algoritmo genético}

El sistema MADE provee un sistema para permitir que el usuario pueda definir su
función fitness de manera cómoda y muy versátil. Para ello, puede editarse el
fichero ``evaluation.properties'' para trabajar con las propiedades descritas
a continuación:

Se permite editar un conjunto de parámetros
necesarios para la ejecución del algoritmo genético: número máximo de
generaciones, tamaño de la población y número de perfiles. Es importante
destacar que el tamaño del cromosoma es directamente proporcional al número de
perfiles.

\begin{verbatim}
# ---
# Global variables
# max allowed evaluation (generations) of the 
# genetic algorithm.
global.MAX_ALLOWED_EVOLUTIONS=30
# population size for the genetic algorithm.
global.POPULATION_SIZE=20
global.NUMBER_OF_PROFILES=1
#---
\end{verbatim}


El sistema MADE permite definir la función de fitness. Para ello se usa un
sistema de scripting sencillo que permite al sistema MADE ejecutar expresiones y
código fuente java de manera muy específica:

\begin{verbatim}
#---
# Labels
# p = number of agents (dead or alive)
# pm = Number of agents that match the state
# a = number of alive agents
# am = number of alive agents that match the state

# growing population
label.alivepopulation.r=.*
label.alivepopulation.c=a>30 && a<90
label.alivepopulation.w=gaussian((double)am, 60.0, 30.0)
#---
\end{verbatim}

Cada arquetipo o característica de la población que desee evaluarse debe estar
presente en el fichero mediante el uso de tres propiedades y debe asociarse de
manera única con una etiqueta:

\begin{description}
 \item[label.(nombre\_de\_etiqueta).r] Propiedad que define la expresión regular
que será buscada en el registro de cada agente. Nota importante: El símbolo
``\\" siempre debe aparecer doble.
 \item[label.(nombre\_de\_etiqueta).c] Propiedad que define la condición
que debe cumplir la población respecto al número de agentes que poseen la
etiqueta. El sistema provee cuatro variables que pueden ser utilizadas:
\begin{description}
 \item[p:] número total de agentes que viven o han vivido durante la ejecución.
 \item[pm:]  número total de agentes que viven o han vivido durante la
ejecución y que tienen la etiqueta.
 \item[a] número de agentes vivos en el momento del fin de la ejecución.
 \item[am] número de agentes  vivos en el momento del fin de la ejecución. y
que tienen la etiqueta.
\end{description}
 \item[label.(nombre\_de\_etiqueta).w] Propiedad que define el peso que consigue
el Entorno MADE ejecutado. Además de proveerse las cuatro variables utilizables
en la anterior propiedad, se ofrece la función \textit{gaussian}, que recibe
como parámetro el valor que se quiere medir, el valor objetivo y la distancia a
éste a partir de la cual el valor debe aportar peso al fitness. Para ponderar
el fitness basta con aplicar un factor multiplicativo.
\end{description}






