

%\begin{savequote}[50mm]
%Historical methodology, as I see it, is a product of common sense applied to circumstances. 
%\qauthor{Samuel E. Morison}
%\end{savequote}


\chapter{Metodología general}
\label{cha:Overall methodology}

La metodología de evaluación del sistema MADE consitirá en el siguiente proceso:

\begin{itemize}
 \item Definición de los parámetros de ejecución del Entorno MADE
 \item Definición de la función fitness
 \item Estudio de la variabilidad del fitness para soluciones aleatorias
 \item Definición de los parámetros de ejecución del algoritmo genético
 \item Ejecución del algoritmo para diferente número de perfiles
 \item Obtención de la mejor solución para los diferentes perfiles
 \item Análisis de los valores obtenidos
\end{itemize}

\section{Definición de los parámetros de ejecución del Entorno MADE}

El entorno MADE se ejecutará con los siguientes parámetros:

\begin{verbatim}
global.NUMBER_OF_PROFILES=1
global.NUMBER_OF_INITIAL_AGENTS=30
global.MAP_DIMENSION=30
global.FOOD=20
global.DAYS=2000

base.BASE_DAYS = 200
base.BASE_ENERGY = 5
base.BASE_SMELL = 3
base.BASE_NUTRITION = 4
base.BASE_BITE = 5
base.BASE_FUR = 5
base.BASE_AGE_TO_BE_ADULT_FEMALE = 42
base.BASE_AGE_TO_BE_ADULT_MALE = 49
base.BASE_PREGNANCY_TIME = 30
\end{verbatim}

\section{Definición de la función Fitness}

Se han utilizado las siguientes etiquetas, condiciones y pesos. 

\begin{verbatim}

# growing population
label.alivepopulation.r=.*
label.alivepopulation.c=a>30 && a<90
label.alivepopulation.w=gaussian((double)am, 60.0, 30.0)

# 1 opressed
label.opressed.r=[\\.\\s]*@DEFENDED[^@]*@DEFENDED[\\.\\s]*
label.opressed.c=am>=a*0.05 && am<=a*0.2
label.opressed.w=gaussian(a==0? 0: (double)am /(double)a, 0.15, 0.15)

# a warrior
label.warrior.r=[\\.\\s]*@NUDGE_OK[\\.\\s]*{5}
label.warrior.c=pm>=a*0.05 && pm<=a*0.2
label.warrior.w=gaussian(p==0? 0: (double)pm /(double)p, 0.15, 0.15)

# an indefense
label.indefense.r=[\\.\\s]*@NUDGED[\\.\\s]*{10}
label.indefense.c=pm>=p*0.05 && pm<=p*0.2
label.indefense.w=gaussian(p==0? 0: (double)pm /(double)p, 0.15, 0.15)

# failed warrior
label.failedwarrior.r=[\\.\\s]*@NUDGE_FAILED[\\.\\s]*{10}
label.failedwarrior.c=pm>=p*0.05 && pm<=p*0.2
label.failedwarrior.w=gaussian(p==0? 0: (double)pm /(double)p, 0.15, 0.15)

\end{verbatim}



\section{Estudio de la variabilidad del fitness}

Se estudiará la variabilidad del fitness para 100 cromosomas, mediante la
ejecución de 100 Entornos MADE utilizando dichos cromosomas.

Utilizando los datos anteriores, se elegirá un número de iteraciones que
ofrezca una media significativa, y se utilizará para calcular el fitness medio
de cada indivíduo de la población del algoritmo genético.


\section{Parámetros de ejecución del algoritmo genético}

\begin{verbatim}
 global.MAX_ALLOWED_EVOLUTIONS=30
 global.POPULATION_SIZE=20
 global.NUMBER_OF_PROFILES={1,2,3,4}
\end{verbatim}
