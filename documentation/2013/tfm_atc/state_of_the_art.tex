
\chapter{Estado del arte}
\label{cha:Estado del arte}


\section{Character Evolution Approach to Generative Storytelling}


\section{Self-organizing Systems}


\section{Artificial societies}



\section{Arquetipos Literarios}

Tal y como indica \cite{web_arquetipos} En la literatura, los lectores encuentran muchos personajes del mismo tipo básico. Las historias y situaciones también son aptas para ser repetidas. Estos patrones predecibles y entendibles universalmente en el arte se llaman arquetipos. La palabra viene del griego y básicamente significa "modelo original". Los escritores usan arquetipos porque los lectores están familiarizados con ellos, estos arquetipos hablan a algo en la conciencia humana y provocan respuestas emocionales. La familiaridad impulsa al lector a continuar la historia.
